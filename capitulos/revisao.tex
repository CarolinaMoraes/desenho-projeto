\chapter{Revisão Bibliográfica}
Nesta seção, os tópicos conceituais e teóricos relacionados ao projeto a ser desenvolvido que irão auxiliar o entendimento do mesmo serão descritos de forma a esclarecer tanto as partes de negócio, como as técnicas. Dessa forma, a compreensão do projeto como todo será mais fácil para o leitor.

\section{Organização financeira}
De acordo com o dicionário Michaelis \cite{Michaelis}, a palavra organização significa preparação de um projeto, com definição de procedimentos e metas. Assim, pode se dizer que organização financeira trata-se de cuidar das finanças (podendo elas ser empresarial, familiar, ou pessoal) afim de atingir um objetivo.

Quando trata-se de finanças pessoais ou familiares, ter uma boa organização financeira significa conhecer quais e quanto são suas despesas e receitas mensais \cite{PlanejamentoFinanceiroFamiliar}. De acordo com o mesmo artigo, é necessário planejar como, quanto e onde a despesa será feita, além de fazer levantamento de preços.

Pode-se dizer então, que as compras, categorizadas como despesas fixas, devido a sua presença frequênte em ambientes pessoais e familiares, precisam ser planejadas e organizadas para impulsionar uma melhor organização financeira entre os envolvidos da compra.

\section{Listas}
As listas de afazeres, comumente chamadas de \textit{to do list} em inglês, é uma lista de lembretes textuais, como por exemplo, "Ir ao médico", "Comprar caixa de 12 ovos", entre outros \cite{Towel}. Essas listas são encontradas em qualquer âmbito, e auxiliam a lembrar de tarefas ou coisas importantes que precisam sobrer uma ação do indívidio, por exemplo, comprar, completar, finalizar.

Seguindo o mesmo príncipio, uma lista de compras se refere à uma lista contendo nomes de produtos nos quais um indivíduo deseja adquirir, podendo conter ou não um limite de gasto total. As listas, além de ser um método eficaz para se organizar e lembrar do que foi escrito, é considerada uma ótima maneira de evitar que alguém ceda à produtos nos quais não quer comprar por serem prejudicias à saúde da pessoa \cite{GroceryList} ou por qualquer outro motivo.

Existem três tipos de compras \cite{ComprasNaoPlanejadas}, sendo que dois tipos podem ter a presença de uma lista de compras:
\begin{enumerate}
\item Completamente planejada, incluindo na lista de compras os produtos e suas respectivas marcas.
\item Parcialmente planejada, incluindo apenas os produtos a serem comprados mas a marca deles serão decididas no ato da compra.
\end{enumerate}

As compras completamente planejadas são consideras com pouco envolvimento emocional do consumidor, enquanto as parcialmente planejadas, apesar de planejadas, podem ser manipuladas por algum critério exterior, como uma promoção \cite{ComprasNaoPlanejadas}. No aplicativo a ser desenvolvido, uma lista de compras pode ser tanto completamente planejada ou parcialmente planejada, de acordo com a necessidade do usuário que está utilizando a plataforma. Uma vez que a lista de compras existe, a compra em si se torna planejada.

\section{Aplicativo \textit{Mobile}}
Um aplicativo \textit{mobile} trata-se de uma aplicação de software que é desenvolvida exclusivamente ou acessável por um dispositivo móvel, como celulares e \textit{smartphones}. O uso dos dispositivos móveis cresceu disparadamente nos últimos anos, e é o principal meio de acesso no Brasil desde 2017 \cite{Celular}.

Atualmente, existem dois sistemas operacionais que são mais utilizados nas plataformas \textit{mobile}: o \textit{Android} e o \textit{IOS}, tratados como concorrentes das marcas \textit{Google} e \textit{Apple}. Os aplicativos desenvolvidos para os dois sistemas operacionais são divididos em dois tipos:
\begin{itemize}
\item Aplicativos nativos.
\item Aplicativos híbridos.
\end{itemize}

Os aplicativos nativos são desenvolvidos utilizando o \label{sig:SDK}\hyperlink{s:SDK}{SDK} do sistema operacional em questão, e não pode ser executados em outros ambientes. Por exemplo, um aplicativo desenvolvido com o SDK da empresa \textit{Apple} não pode ser executado em ambiente \textit{Android} e vice-versa \cite{MobileApps}.

Já os aplicativos híbridos são desenvolvidos utilizando tecnologias web, como \label{sig:HTML}\hyperlink{s:HTML}{HTML}, \label{sig:CSS}\hyperlink{s:CSS}{CSS} e \textit{Javascript}, e por isso, podem ser executados em qualquer sistema operacionais móvel através de um container nativo. Apesar de poder ser executado em diversos dispositivos através de apenas um código, os aplicativos híbridos	 possuem certa limitações ao acessar recursos nativos do dispositivo, como câmeras, microfones e sistemas internos de armazenamento e memória, porém, atualmente já existem soluções que facilitam a comunicação entre aplicativo híbrido e o dispositivo, como o \textit{React Native} \cite{MobileApps}.




