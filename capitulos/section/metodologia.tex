\section{Metodologia de Gestão e Desenvolvimento de Projeto}
O grupo optou pelo uso de uma metodologia ágil para o projeto e, após algumas reuniões, foi decidido que o que guiaria o processo de desenvolvimento seria uma junção de frameworks extremamente úteis para a melhor performance do time, o Scrum e o Kanban. 
O Scrum se baseia na divisão do projeto em vários ciclos de atividades - conhecidos como Sprints - com diversas cerimônias e interações frequentes da equipe a fim de alinhar o que tem sido feito, tratar impedimentos e pensar em maneiras de otimizar o processo de trabalho, aumentando a agilidade no desenvolvimento. Já o Kanban é um método de controle e gestão de forma visual, geralmente feito com o uso de post-its coloridos que reforçam a simbologia das tarefas e ações que precisam ser feitas, estão sendo feitas, ou foram concluídas. 

Dentro das metodologias ágeis é muito comum - e essencial - a definição do Product Backlog contendo todas as funcionalidades desejadas na aplicação pelo cliente e elencadas por prioridade. Esse backlog é, posteriormente, dividido em tarefas que serão distribuídas em cada Sprint, gerando o Sprint Backlog. É nesse momento em que, no projeto, o Kanban será aplicado. Dessa forma, o Scrum será utilizado para a determinação do método de desenvolvimento iterativo e incremental e o Kanban será utilizado para métricas e fluxos de produção da equipe.