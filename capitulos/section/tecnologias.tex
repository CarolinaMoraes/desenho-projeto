\section{Tecnologias Utilizadas}

As tecnologias que decidimos utilizar foram escolhidas a partir do conhecimento prévio da equipe, da curva de aprendizado e levando em consideração também o tamanho das comunidades que já a utilizam, visando um maior apoio e material de pesquisa.
Dito isso, escolhemos as seguintes tecnologias:

\subsection{Linguagens}

\subsubsection{Back-end}

Decidimos que a linguagem para o \gls{backend} seria o Java. A linguagem se adequa à nossa proposta e atende o paradigma de linguagem orientada a objetos do qual nos foi orientado a utilizar. 
A comunidade de Java é extensa e ativa, contribuindo com muitos materiais e recursos. Ainda podemos destacar que a utilização da linguagem previamente pelos integrantes da equipe também foi impactante na consolidação dessa decisão.

\subsubsection{Mobile}
Para o desenvolvimento da plataforma mobile decidimos utilizar o Javascript. A linguagem possui também uma comunidade ativa e uma variedade de materiais disponíveis e atualizados. Apesar de nem todos os integrantes terem tido contato prévio, por conta da facilidade de assimilação e necessidade de poucos recursos para a configuração do ambiente de desenvolvimento, optamos pelo Javascript.

\subsection{Frameworks e ORMs}

\subsubsection{Back-end}

Para o \gls{backend} decidimos utilizar o \gls{framework} Spring, usando a ferramenta Spring Boot que proporciona agilidade na criação das aplicações pois segue a filosofia de Convention over Configuration\cite{Devopedia2020}, nos poupando de depreender muito tempo nas configurações. Não obstante, a \gls{framework} facilita o desenvolvimento pois nos propicia a utilização de módulos que julgarmos necessários (como Spring MVC e Spring Data JDBC). Além disso, há uma gama vasta de materiais para consultarmos.
Como ferramenta \label{sig:ORM}\hyperlink{s:ORM}{ORM} decidimos usar o Hibernate pela consolidação dele no mercado e o uso amplo em aplicações Java que necessitam de mapeamento relacional dos dados. Por conta da quantidade de modelos da aplicação, julgamos necessário utilizar uma ferramenta que facilitasse esse processo. 

\subsubsection{Mobile}
Na aplicação mobile decidimos utilizar o \gls{framework} React-Native. Esta \gls{framework} gera aplicativos nativos, não necessita de muitos recursos e configurações para montar o ambiente de desenvolvimento e possibilita um conforto maior no desenvolvimento do código por ser uma \gls{framework} Javascript. Não obstante, também é uma \gls{framework} com larga quantidade de recursos para consulta além de uma comunidade muito ativa.

\subsection{Banco de dados}
O banco de dados que escolhemos foi o MySQL pois precisávamos para a nossa proposta de um banco de dados relacional e que fosse possível de ser hospedado na AWS. Verificamos que o MySQL cobria não apenas esses critérios mas também possui uma ferramenta gráfica (MySQL Workbench) que facilita a visualização e a operação do banco. Além disso os integrantes da equipe já tiveram experiências com a ferramenta anteriormente.

\subsection{Gerenciamento de tarefas}
Para o gerenciamento das tarefas optamos pela ferramenta Trello por ser gratuita, de fácil manuseio e visualização.
Além disso, o Trello figura entre as ferramentas que foram utilizadas com sucesso nos semestres anteriores durante o desenvolvimento de projetos.

\subsection{Versionamento}
Para o controle de versão do desenvolvimento da aplicação optamos pelo Git com repositório no Github. Esta escolha foi realizada devido à experiência prévia da equipe com a ferramenta e pela quantidade de recursos oferecidos pela plataforma na gestão do desenvolvimento de aplicações.
Para o versionamento do projeto utilizamos o Apache Subversion (SVN). Concentramos no repositório fornecido pelos professores todas as entregas previstas (incluindo as versões atualizadas dos códigos do repositório externo do Github).



%%% Local Variables:
%%% mode: latex
%%% TeX-master: "../proposta"
%%% End:
