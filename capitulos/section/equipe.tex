\section{Organização da Equipe}
As primeiras reuniões de alinhamento entre os membros da equipe foram essenciais para que os integrantes pudessem interagir e compartilhar mais sobre quais conhecimentos prévios possuem, quais são suas áreas de maior facilidade e interesse e dessa forma planejar uma estratégia de desenvolvimento que, apesar de desafiadora, também seja inclusiva a todos do grupo. 

No Scrum, metodologia escolhida para o projeto, a definição de alguns papéis é necessária. São eles: 

\begin {itemize}
\item \textbf{Product Owner}: É o responsável pelos interesses do cliente no projeto, captando, interpretando e repassando ao time de desenvolvimento suas necessidades. Quem assumiu este papel foi o Fabio Mendes;

\item \textbf{Scrum Master}: É o facilitador no desenvolvimento do projeto, cuidando para que as cerimônias do Scrum sejam devidamente cumpridas e ajudando a resolver os impedimentos que possam atrapalhar a equipe de desenvolvimento. Quem assumiu este papel foi a Carolina de Moraes;

\item \textbf{Development Team}: Time que cuida do desenvolvimento técnico do projeto. Quem assumiu este papel foram Alkindar Rodrigues, Gabriely Bicigo, Leonardo Naoki e Mariana Zangrossi.

\end {itemize}

É importante salientar que, apesar da divisão de papéis, todos os participantes estão contribuindo no desenvolvimento com o objetivo de entregar o projeto na data estimada.

Para uma divisão mais organizada e formal de tarefas, foi montada uma tabela de responsabilidades dos integrantes referentes aos tópicos de desenvolvimento do projeto.

\begin{center}
\begin{tabular}{|l|c|c|c|c|c|c|}
\hline
\multicolumn{1}{|c|}{\textbf{Atividade}} & \textbf{Alkindar}     & \textbf{Carolina}     & \textbf{Fabio}        & \textbf{Gabriely}     & \textbf{Leonardo}     & \textbf{Mariana}      \\ \hline
Back-End                                 & x                     & \multicolumn{1}{l|}{} & x                     & \multicolumn{1}{l|}{} & x                     & \multicolumn{1}{l|}{} \\ \hline
Banco de dados                           & x                     & \multicolumn{1}{l|}{} & x                     & \multicolumn{1}{l|}{} & x                     & \multicolumn{1}{l|}{} \\ \hline
Blog                                     & x                     & x                     & x                     & x                     & x                     & x                     \\ \hline
Documentação                             & x                     & x                     & x                     & x                     & x                     & x                     \\ \hline
Front-End                                & \multicolumn{1}{l|}{} & x                     & \multicolumn{1}{l|}{} & x                     & \multicolumn{1}{l|}{} & x                     \\ \hline
Vídeos                                   & \multicolumn{1}{l|}{} & x                     & \multicolumn{1}{l|}{} & x                     & \multicolumn{1}{l|}{} & x                     \\ \hline
\end{tabular}

\caption{Tabela de responsabilidades}\label{tab:tab_resp}
\end{center}
