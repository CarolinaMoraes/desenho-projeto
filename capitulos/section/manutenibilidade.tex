\section{Manutenibilidade da aplicação}

É fundamental para o desenvolvimento do projeto, tanto o previsto
quanto em avanços posteriores, que a aplicação atinja um nível adequado
de qualidade, e para tanto certos requisitos de manutenibilidade devem
ser estabelecidos.  Estes requisitos permitem a

\subsection{\emph{Logs}}

Como forma de monitorar a aplicação em tempo de execução,
especialmente na camada de servidor, \emph{logs} serão usados para
registrar o estado dos objetos. A ferramenta Log4j será usada, uma vez
que os membros do time já tem mais familiaridade com ela. Esta
ferramenta perimite o registro em diversos níveis, como
\begin{itemize}
  \em
\item info
\item debug
\item warn
\item error
\end{itemize}
e pode ser configurada para que apenas os dois últimos
sejam registrados no ambiente de produção. Desta forma a cada ponto de
falha da aplicação um log de nível apropriado será colocado para que
problemas sejam rapidamente identificados, analisados e resolvidos.

\subsection{Integração Contínua}

Visando manter o serviço sempre atualizado para o usuário, a
ferramenta de integração contínua Jenkins foi selecionada para a
implantação da aplicação \gls{backend} em produção. Ela
permite, a partir do código no repositório git, a execução de testes,
o build e o \emph{deploy} para o ambiente de produção, automatizando tarefas
complicadas de se executar em máquinas remotas.

Para o \gls{frontend}, no entanto, não existem ferramentas de
integração contínua, uma vez que as imagens do aplicativo mobile
precisam ser aprovadas pelas lojas de aplicativo. Contudo, ainda é
possível usar ferramentas de CI para a execução de testes e
verificação da integridade do código a cada nova versão.

\subsection{\emph{Code Conventions}}

As convenções de código são acordos internos ao time que visam
estandartizar a forma como os diversos integrantes do time produzem
seus códigos.  Elas visam facilitar o entendimento mútuo entre os
integrantes do time, de modo o estilo de progamação seja indistiguível
e independente de seus autores.  Geralmente, as convenções de código
estabelecem estilos para se organizar o código textualmente, isto é,
dizem respeito a forma como nomes de variáveis são escolhidas e
comentários são posicionados, por exemplo.

As convenções adotadas são baseadas na especificação da
\citeauthor{Oracle1997}, de 1997. Esta é comumente usadas para o
desenvolvimento na linguagem Java, e muito próxima do padão adotado em
JavaScript, e vale destacar o seguintes pontos:
\begin{itemize}
\item Minimizar o uso de variáveis, funções e objetos globais.
\item As declarações globais estarão preferencialmente no início do arquivo.
\item Declarar as variáveis próximo do ponto onde elas serão inicializadas.
\item A indentação é de 4 espaços.
\item Linhas mais longas que 80 caracteres serão quebradas e
  indentadas a 8 espaços.
\item Pacotes e variáveis com nomes curtos, em \texttt{camelCase} e substantivos.
\item Classes e interfaces em \texttt{CamelCase} e substantivos.
\item Métodos em \texttt{camelCase} e verbos.
\item Constantes em \texttt{UPPER\_CASE}.
\end{itemize}

No entanto, especificamente para a linguagem Java,
\begin{itemize}
\item Classes e métodos devem ser documentados com um comentário na
  seguinte forma, uma vez que as IDEs reconhecem este formato e
  formatam o text na forma de \emph{pop-ups} quando o cursor está
  sobre uma referência a esta classe.
\begin{verbatim}
/**
 * Class ListService
 *
 * Implementar endpoints para as funcionalidades de lista.
 */
\end{verbatim}
\end{itemize}

\subsection{Testes}

Testes são uma ferramenta fundamental para o desenvolvimento da
aplicação, uma vez que garenatem, em tempo de compilação, o
comportamento correto do aplicativo. Para além disso, testes tem um
papel de documentação, uma vez que encapsulam de forma breve o
comportamento esperado das classes e métodos produzidos, e podem ser
consultados em caso de dúvida quando ao uso destes. Este tipo de teste
é chamado de teste unitário, em oposição aos testes de integração, que
verificam o funcionamento da aplicação de ponta-a-ponta, isto é, a
partir de uma chamada a um endpoint, apenas os serviços externos são
mimetizados, garantindo o funcionamento correto de toda a aplicação.

Desta forma, a construção de testes, de ambos os tipos é de extrema
importância para a elaboração do projeto visando a sua
manutenibilidade, e será o primeiro passo de uma sprint, após o
planejamento, a construção de testes relevantes para a tarefa,
seguindo os princípios do TDD. Como as ferramentas de teste são
específicas de cada linguagem, cada camada da aplicação fará uso de
frameworks distintos.

O \gls{backend} será testado com o framework JUnit, fazendo uso da
biblioteca Mockito quando necessário simular comportamentos de objetos
que não são o alvo da suite de teste. Este framework ainda oferece
ferramentas para testar o banco de dados, ou melhor, testar a conexão
com o banco e verificar o comportamento das classes de acesso a ele.
Já os teste de \gls{frontend} serão feitos com a biblioteca Jest,
que auxilia a construção de testes unitários, e por se tratar de uma
GUI, as interções de usuário devem ser simuladas também. Para tanto, a
blioteca React Testing Library será usada.

%%% Local Variables:
%%% mode: latex
%%% TeX-master: "../../desenho"
%%% End:
