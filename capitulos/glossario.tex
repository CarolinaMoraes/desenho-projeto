\newglossaryentry{API}{
  name={API},
  description={Uma Interface de Progamação de Aplicação (
    \emph{Application Programming Interface}), são pontos que a
    aplicação expoõe para permitir que usuários ou serviços externos
    executem tarefas dentro da aplicação.}
}

\newglossaryentry{deploy}{
  name={deploy},
  description={Processo pelo qual a aplicação é implantada em ambiente
  de prpdução, e está disponível para os usuários finais.}
}

\newglossaryentry{frontend}{
  name={front-end},
  description={Um sistema \emph{front-end} é aquele que encontra na
    camada cliente, em uma aplicação de duas camadas. Sua
    principal função, no escopo deste projeto, é atuar como interface
    gráfica para o usário, coletar dados e enviá-los par o \emph{back-end}.
  }
}

\newglossaryentry{GUI}{
  name={GUI},
  description={Interface gráfica de usuário é uma forma visual de se
    apresentar dados e coletar interações com o usuário, em oposição a
    linha de comando, que funciona por texto apenas.}
}

\newglossaryentry{backend}{
  name={back-end},
  description={Um sistema \emph{back-end} é aquele que encontra na
    camada de servidor, em uma aplicação de duas camadas. Sua
    principal função é fornecer informações e capacidade de
    processamento a aplicação cliente.
  }
}

\newglossaryentry{daily}{
  name={daily},
  description={Cerimônia diária do Scrum, é uma reunião de geralmente 15 minutos com o objetivo de alinhar o que será feito no dia e se existe algum impedimento.}}

\newglossaryentry{REST}{
  name={REST},
  description={A Tranfêrencia por Representação de Estado é uma forma
    de se transferir dados na qual os atirbutos de um objeto (seu
    estado) são serializados em um arquivo de formato específico, e
    o objeto pode ser reconstituído na aplicação que recebe o arquivo.}
}

\newglossaryentry{kanban}{
  name={kanban},
  description={Framework de gestão visual de fluxo de produção, que consiste no uso de cartões coloridos que separam as diferentes fases do desenvolvimento das tarefas (to do, doing, done).}
}

\newglossaryentry{scrum}{
  name={scrum},
  description={Framework ágil utilizado em desenvolvimentos iterativos
    e incrementais.}
}

\newglossaryentry{sprint}{
  name={sprint},
  description={Período de tempo limitado dentro do Scrum onde uma quantidade de histórias de usuário incrementáveis são desenvolvidas.}
}

\newglossaryentry{sprint-backlog}{
  name={sprint backlog},
  description={Lista das histórias de usuário que serão desenvolvidas durante a Sprint a ser iniciada.}
}

\newglossaryentry{product-backlog}{
  name={product backlog},
  description={ Lista de todos os requisitos e funcionalidades desejadas para um produto.}
}

\newglossaryentry{framework}{
  name={framework},
  description={
    Conjunto de ações e estratégias que possuem um ojeto e objetivo em específico.\\
    Na programação, é um conjunto de pacotes e bibliotecas que abstarai
    alguma função complexa, geralmente de nível mais baixo, e sobre o
    qual uma aplicação pdoe ser contruída. Ex: o framework web Spring
    asbtrai a lógica de implementação de um servidor, auxiliando a
    criação de endpoints, rotas e processamento de HTTP.}}

\newglossaryentry{planning}{
  name={sprint planning},
  description={Cerimônia do Scrum que ocorre antes do início
    de uma nova Sprint com o objetivo de planejar quais histórias
    de usuário serão desenvolvidas naquela iteração.
  }
}

\newglossaryentry{review}{
  name = {sprint review},
  description= {
    Cerimônia do Scrum que ocorre no fim da Sprint
    com o objetivo de validar as entregas daquele período.
  }
}

\newglossaryentry{retrospective}{
  name = {sprint retrospective},
  description = {
    Cerimônia do Scrum que ocorre no fim da Sprint com
    o objetivo de avaliar os pontos positivos e negativos
    da Sprint, agrupando possibilidades de melhorias
    para a próxima Sprint.
  }
}

%%% Local Variables:
%%% mode: latex
%%% TeX-master: "../desenho"
%%% End:
