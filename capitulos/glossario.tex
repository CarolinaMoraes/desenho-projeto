\newglossaryentry{API}{
  name={API},
  description={Uma Interface de Progamação de Aplicação (
    \emph{Application Programming Interface}), são pontos que a
    aplicação expoõe para permitir que usuários ou serviços externos
    executem tarefas dentro da aplicação.}
}

\newglossaryentry{deploy}{
  name={deploy},
  description={Processo pelo qual a aplicação é implantada em ambiente
  de prpdução, e está disponível para os usuários finais.}
}

\newglossaryentry{frontend}{
  name={front-end},
  description={Um sistema \emph{front-end} é aquele que encontra na
    camada cliente, em uma aplicação de duas camadas. Sua
    principal função, no escopo deste projeto, é atuar como interface
    gráfica para o usário, coletar dados e enviá-los par o \emph{back-end}.
  }
}

\newglossaryentry{GUI}{
  name={GUI},
  description={Interface gráfica de usuário é uma forma visual de se
    apresentar dados e coletar interações com o usuário, em oposição a
    linha de comando, que funciona por texto apenas.}
}

\newglossaryentry{backend}{
  name={back-end},
  description={Um sistema \emph{back-end} é aquele que encontra na
    camada de servidor, em uma aplicação de duas camadas. Sua
    principal função é fornecer informações e capacidade de
    processamento a aplicação cliente.
  }
}


\newglossaryentry{REST}{
  name={REST},
  description={A Tranfêrencia por Representação de Estado é uma forma
    de se transferir dados na qual os atirbutos de um objeto (seu
    estado) são serializados em um arquivo de formato específico, e
    o objeto pode ser reconstituído na aplicação que recebe o arquivo.}
}

\newglossaryentry{framework}{
  name={framework},
  description={Um conjunto de pacotes e bibliotecas que abstarai
    alguma função complexa, geralmente de nível mais baixo, e sobre o
    qual uma aplicação pdoe ser contruída. Ex: o framework web Spring
    asbtrai a lógica de implementação de um servidor, auxiliando a
    criação de endpoints, rotas e processamento de HTTP.}
}


%%% Local Variables:
%%% mode: latex
%%% TeX-master: "../desenho"
%%% End:
