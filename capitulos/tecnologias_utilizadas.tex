\chapter{Tecnologias Utilizadas}

As tecnologias que decidimos utilizar foram escolhidas a partir do conhecimento prévio da equipe, da curva de aprendizado e levando em consideração também o tamanho das comunidades que já a utilizam, visando um maior apoio e material de pesquisa.
Dito isso, escolhemos as seguintes tecnologias:

\section{Linguagens}

\subsection{Back-end}

Decidimos que a linguagem para o back-end seria o Java. A linguagem se adequa à nossa proposta e atende o paradigma de linguagem orientada a objetos do qual nos foi orientado a utilizar. 
A comunidade de Java é extensa e ativa, contribuindo com muitos materiais e recursos e além disso podemos destacar que a utilização da linguagem previamente pelos integrantes da equipe também foi impactante na consolidação dessa decisão.

\subsection{Mobile}
Para o desenvolvimento da plataforma mobile decidimos utilizar o Javascript. A linguagem possui também uma comunidade ativa e uma variedade de materiais disponíveis e atualizados. Apesar de nem todos os integrantes terem tido contato prévio, por conta da facilidade de assimilação e necessidade de poucos recursos para a configuração do ambiente de desenvolvimento, optamos pelo Javascript.

\section{Frameworks e ORMs}

\subsection{Back-end}

Para o back-end decidimos utilizar o framework Spring, usando a ferramenta Spring Boot que proporciona agilidade na criação das aplicações pois segue a filosofia de Convention over Configuration\cite{Devopedia2020}, nos poupando de depreender muito tempo nas configurações. Não obstante, a framework facilita o desenvolvimento pois nos propicia a utilização de diversos módulos que julgarmos necessários. Como Spring MVC e Spring Data JDBC. Além disso, há uma gama vasta de materiais para consultarmos.
Como ferramenta ORM decidimos usar o Hibernate pela consolidação dele no mercado e o uso amplo em aplicações Java que necessitam de mapeamento relacional dos dados. Por conta da quantidade de modelos da aplicação, julgamos necessário utilizar uma ferramenta que facilitasse esse processo. 

\subsection{Mobile}
Na aplicação mobile decidimos utilizar o framework React-Native. Esta framework gera aplicativos nativos, não necessita de muitos recursos e configurações para montar o ambiente de desenvolvimento e possibilita um conforto maior no desenvolvimento do código por ser uma framework Javascript. Não obstante, também é uma framework com larga quantidade de recursos para consulta além de uma comunidade muito ativa.

\section{Banco de dados}
O banco de dados que escolhemos foi o MySQL pois precisávamos para a nossa proposta de um banco de dados relacional e que fosse possível de ser hospedado no Heroku. Verificamos que o MySQL cobria não apenas esses critérios mas também possui uma ferramenta gráfica (MySQL Workbench) que facilita a visualização e a operação do banco e além disso os integrantes da equipe já tiveram experiências com a ferramenta anteriormente.

\section{Gerenciamento de tarefas}
Para o gerenciamento das tarefas optamos pela ferramenta Trello por ser gratuita, de fácil manuseio e visualização.
Além disso, a ferramenta figura entre as ferramentas que foram utilizadas com sucesso nos semestres anteriores em Projetos.



%%% Local Variables:
%%% mode: latex
%%% TeX-master: "../proposta"
%%% End:
